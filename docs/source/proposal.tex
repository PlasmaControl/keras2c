\documentclass{article}
\usepackage[letterpaper, portrait, margin=1in]{geometry}
\usepackage[utf8]{inputenc}
\usepackage{amsfonts,amsmath,amssymb,amsthm}
\usepackage{bm,nicefrac}
\usepackage{graphicx}
\usepackage{hyperref}
\usepackage{subcaption}
\usepackage[section]{placeins}
\usepackage{textgreek}
\usepackage{changepage}
\usepackage{authblk}

\title{Keras2c: A simple library for converting Keras neural networks
to real-time friendly C}
\author[1]{Rory Conlin}
\author[2]{Keith Erickson}
\author[3]{Joe Abbate}
\author[1,2]{Egemen Kolemen}
\affil[1]{Dept. of Mechanical \& Aerospace Engineering, Princeton University}
\affil[2]{Princeton Plasma Physics Laboratory}
\affil[3]{Dept. of Astrophysical Sciences, Princeton University}


\begin{document}

\maketitle


\section*{Abstract}\label{abstract}

With the growth of machine learning models and neural networks in
measurement and control systems comes the need to deploy these models in
a way that is compatible with existing systems. Existing options for
deploying neural networks either introduce very high latency, requires
expensive and time consuming work to integrate into existing code bases,
or only support a very limited subset of model types. We have therefore
developed a new method, called Keras2c, which is a simple library for
converting Keras/TensorFlow neural network models into real time
compatible C code. It supports a wide range of Keras layer and model
types, including multidimensional convolutions, recurrent layers, well
as multi-input/output models, and shared layers. Keras2c re-implements
the core components of Keras/TensorFlow required for predictive forward
passes through neural networks in pure C, relying only on standard
library functions. The core functionality consists of only
\textasciitilde{}1200 lines of code, making it extremely lightweight and
easy to integrate into existing codebases. Keras2c has been sucessfully
tested in experiments and is currently in use on the plasma control
system at the DIII-D National Fusion Facility at General Atomics in San
Diego.


\section*{Motivation}\label{motivation}

TensorFlow is one of the most popular libraries for developing and
training neural networks, and contains a high level Python API called
Keras that has become extremely popular due to its ease of use and rich
feature set. As the use of machine learning and neural networks grows in
the field of diagnostic and control systems, one of the central
challenges remains how to deploy the resulting trained models in a way
that can be easily integrated into existing systems, particularly for
real time predictions using machine learning models. Given that most
machine learning development traditionally takes place in Python, most
deployment schemes involve calling out to a Python process (often
running on a distant network connected server) and using the existing
Python libraries to pass data through the model. This introduces large
latency, and is generally not feasible for real time applications. Other
options include rewriting the entire network using the existing
TensorFlow C/C++ API, though this is extremely time consuming, and
requires linking the resulting code against the full TensorFlow library,
containing millions of lines of code and with a binary size up to
several GB. The release of TensorFlow 2.0 contained a new possibility,
called "TensorFlow Lite", a reduced library designed to run on mobile
and IoT devices. However, TensorFlow Lite only supports a very limited
subset of the full Keras API. Therefore, we present a new option,
Keras2c, a simple library for converting Keras/TensorFlow neural network
models into real time compatible C code.


\section*{Method}\label{method}



Keras2c consists of two primary components: a backend library of C
functions that each implement a single layer of a neural net (eg, Dense,
Conv2D, LSTM), and a Python script that generates C code to call the
layer functions in the right order to implement the network. The total
library of backend layer functions is only $\sim$1200 lines
of code, and uses only C standard library functions, yet covers a very
wide range of Keras functionality, summarized below:



\subsubsection*{Supported Functionality}\label{supported-layers}

\begin{itemize}
  \setlength\itemsep{0em}
\item
  \textbf{Core Layers}: Dense, Activation, Flatten, Input, Reshape, Permute, RepeatVector
\item
  \textbf{Convolution Layers}: Convolution (1D/2D/3D, with arbitrary stride/dilation/padding), Cropping (1D/2D/3D), UpSampling (1D/2D/3D), ZeroPadding (1D/2D/3D)
\item
  \textbf{Pooling Layers}: MaxPooling (1D/2D/3D), AveragePooling (1D/2D/3D), GlobalMaxPooling (1D/2D/3D), GlobalAveragePooling (1D/2D/3D)
\item
  \textbf{Recurrent Layers}: SimpleRNN, GRU, LSTM (statefull or stateless)
\item
  \textbf{Embedding Layers}: Embedding
\item
  \textbf{Merge Layers}: Add, Subtract, Multiply, Average, Maximum, Minimum, Concatenate, Dot
\item
  \textbf{Normalization Layers}: BatchNormalization
\item
  \textbf{Layer Wrappers}: TimeDistributed, Bidirectional
\item
  \textbf{Activations}: ReLU, tanh, sigmoid, hard sigmoid, exponential, softplus, softmax, softsign, LeakyReLU, PReLU, ELU, ThresholdedReLU
\end{itemize}



The Keras2c Python script takes in a trained Keras model and extracts
the weights and other parameters, and parses the graph structure to
determine the order that functions should be called to obtain the
correct results. It then generates C code for a predictor function, that
can be called with a set of inputs to generate predictions. It also
generates helper functions for initializing and cleanup, to handle
memory allocation (by default all variables are declared on the stack,
though it also supports the option of dynamically allocating memory
before execution). In addition to simple sequential models, Keras2c also
supports more complicated architectures created using the Keras
functional API, including multi-input/multi-output networks with
complicated branching and merging internal structures.
\begin{figure}[h!]
\centering
\includegraphics[width=3.5in]{flow_graph.png}
\caption{Workflow of converting Keras model to C code with Keras2C}
\end{figure}

To confirm that the generated code accurately reproduces the outputs of
the original model, Keras2c also generates sample input/output pairs
from the original network. It then automatically tests the generated
code with the same inputs to verify that the generated code produces
equivalent outputs.


\section*{Benchmarks}\label{benchmarks}

Keras2c has also been benchmarked against Python Keras/TensorFlow for
single CPU performance, and the generated code has been shown to be
significantly faster for small to medium sized models. (All tests
conducted on Intel Core i7-8750H CPU @ 2.20GHz, single threaded, 32GB
RAM. Keras2c compiled with GCC 7.4.0 with -O3 optimization. Python Keras
v2.2.4, TensorFlowCPU v1.13.1, mkl v2019.1)

\begin{figure}[h]
\centering
\includegraphics[width=6in]{benchmarking.png}
\caption{Benchmarking results, Keras2c vs Keras/Tensorflow in Python.}
\end{figure}

\end{document}